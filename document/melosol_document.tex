\PassOptionsToPackage{unicode=true}{hyperref} % options for packages loaded elsewhere
\PassOptionsToPackage{hyphens}{url}
%
\documentclass[english,man]{apa6}
\usepackage{lmodern}
\usepackage{amssymb,amsmath}
\usepackage{ifxetex,ifluatex}
\usepackage{fixltx2e} % provides \textsubscript
\ifnum 0\ifxetex 1\fi\ifluatex 1\fi=0 % if pdftex
  \usepackage[T1]{fontenc}
  \usepackage[utf8]{inputenc}
  \usepackage{textcomp} % provides euro and other symbols
\else % if luatex or xelatex
  \usepackage{unicode-math}
  \defaultfontfeatures{Ligatures=TeX,Scale=MatchLowercase}
\fi
% use upquote if available, for straight quotes in verbatim environments
\IfFileExists{upquote.sty}{\usepackage{upquote}}{}
% use microtype if available
\IfFileExists{microtype.sty}{%
\usepackage[]{microtype}
\UseMicrotypeSet[protrusion]{basicmath} % disable protrusion for tt fonts
}{}
\IfFileExists{parskip.sty}{%
\usepackage{parskip}
}{% else
\setlength{\parindent}{0pt}
\setlength{\parskip}{6pt plus 2pt minus 1pt}
}
\usepackage{hyperref}
\hypersetup{
            pdftitle={The MeloSol Corpus},
            pdfkeywords={corpus studies, FAIR data, kern},
            pdfborder={0 0 0},
            breaklinks=true}
\urlstyle{same}  % don't use monospace font for urls
\usepackage{graphicx,grffile}
\makeatletter
\def\maxwidth{\ifdim\Gin@nat@width>\linewidth\linewidth\else\Gin@nat@width\fi}
\def\maxheight{\ifdim\Gin@nat@height>\textheight\textheight\else\Gin@nat@height\fi}
\makeatother
% Scale images if necessary, so that they will not overflow the page
% margins by default, and it is still possible to overwrite the defaults
% using explicit options in \includegraphics[width, height, ...]{}
\setkeys{Gin}{width=\maxwidth,height=\maxheight,keepaspectratio}
\setlength{\emergencystretch}{3em}  % prevent overfull lines
\providecommand{\tightlist}{%
  \setlength{\itemsep}{0pt}\setlength{\parskip}{0pt}}
\setcounter{secnumdepth}{0}

% set default figure placement to htbp
\makeatletter
\def\fps@figure{htbp}
\makeatother

% Manuscript styling
\usepackage{upgreek}
\captionsetup{font=singlespacing,justification=justified}

% Table formatting
\usepackage{longtable}
\usepackage{lscape}
% \usepackage[counterclockwise]{rotating}   % Landscape page setup for large tables
\usepackage{multirow}		% Table styling
\usepackage{tabularx}		% Control Column width
\usepackage[flushleft]{threeparttable}	% Allows for three part tables with a specified notes section
\usepackage{threeparttablex}            % Lets threeparttable work with longtable

% Create new environments so endfloat can handle them
% \newenvironment{ltable}
%   {\begin{landscape}\begin{center}\begin{threeparttable}}
%   {\end{threeparttable}\end{center}\end{landscape}}
\newenvironment{lltable}{\begin{landscape}\begin{center}\begin{ThreePartTable}}{\end{ThreePartTable}\end{center}\end{landscape}}

% Enables adjusting longtable caption width to table width
% Solution found at http://golatex.de/longtable-mit-caption-so-breit-wie-die-tabelle-t15767.html
\makeatletter
\newcommand\LastLTentrywidth{1em}
\newlength\longtablewidth
\setlength{\longtablewidth}{1in}
\newcommand{\getlongtablewidth}{\begingroup \ifcsname LT@\roman{LT@tables}\endcsname \global\longtablewidth=0pt \renewcommand{\LT@entry}[2]{\global\advance\longtablewidth by ##2\relax\gdef\LastLTentrywidth{##2}}\@nameuse{LT@\roman{LT@tables}} \fi \endgroup}

% \setlength{\parindent}{0.5in}
% \setlength{\parskip}{0pt plus 0pt minus 0pt}

% \usepackage{etoolbox}
\makeatletter
\patchcmd{\HyOrg@maketitle}
  {\section{\normalfont\normalsize\abstractname}}
  {\section*{\normalfont\normalsize\abstractname}}
  {}{\typeout{Failed to patch abstract.}}
\makeatother
\shorttitle{MeloSol}
\author{David John Baker\textsuperscript{1}}
\affiliation{
\vspace{0.5cm}
\textsuperscript{1} Louisiana State University}
\authornote{David John Baker now works at Flatiron School in London, England.


Correspondence concerning this article should be addressed to David John Baker, . E-mail: davidjohnbaker1@gmail.com}
\keywords{corpus studies, FAIR data, kern\newline\indent Word count: X}
\DeclareDelayedFloatFlavor{ThreePartTable}{table}
\DeclareDelayedFloatFlavor{lltable}{table}
\DeclareDelayedFloatFlavor*{longtable}{table}
\makeatletter
\renewcommand{\efloat@iwrite}[1]{\immediate\expandafter\protected@write\csname efloat@post#1\endcsname{}}
\makeatother
\usepackage{lineno}

\linenumbers
\usepackage{csquotes}
\ifnum 0\ifxetex 1\fi\ifluatex 1\fi=0 % if pdftex
  \usepackage[shorthands=off,main=english]{babel}
\else
  % load polyglossia as late as possible as it *could* call bidi if RTL lang (e.g. Hebrew or Arabic)
  \usepackage{polyglossia}
  \setmainlanguage[]{english}
\fi

\title{The MeloSol Corpus}

\date{}

\abstract{
This paper introduces the \emph{MeloSol} corpus, a collection of 783 Western, tonal monophonic melodies. We first begin by describing the overal structure of the corpus, then proceede to detail its contents as they would be helpful for researchers working in the field of computational musicology or music psychology. In order to contextualizs the MeloSol corpus, compare descriptive statistics generated using the FANTASTIC feature extraction toolkit with that of the Essen Folk Song Collection as well as The Densmore Collection of Native American Songs. We suggest posible uses of this corpus including extending research which investigates Western tonality, perceptual experiments neededing novel ecological stimuli, or work involving the musical generation of monophonic melodies in the style of Western tonal.
}

\begin{document}
\maketitle

\hypertarget{introdution}{%
\section{Introdution}\label{introdution}}

This data report introduces the \emph{MeloSol} corpus, a collection of 783 monophonic melodies taken from \emph{A New Approach to Sight Singing: Fifth Edition} (Berkowitz, Fontrier, Kraft, Goldstein, \& Smaldone, 2011).
The title \emph{MeloSol} derives from a combination of the corpus' content-- \emph{Mel} odic data-- and the first name of the original author of the collection, \emph{Sol} Berkowitz.

The corpus is divided into two major sections: a collection of sight singing melodies composed specifically for pedagogical purposes (n = XXX) and examples from the Western Classical Music canon (n = XXX).

\begin{itemize}
\tightlist
\item
  Point of Edit
\end{itemize}

Within each of the two larger sections exists FIVE further subdivisions.
These five subdivisions tend to be mapped in conjunction with aural skills classroom.
For example, the first section of both the sight singing melodies and the first section of the Literature align with melodies that a first semester undergraduate student would be expected to learn in their first semester of college in an aural skills classroom.
Each section is meant to increase in difficulty.
The fifth and final section of both the sight singing melodies and examples from the literature contains melodies either meant to be atonal or have some sort of unstable tonality (bi-tonality/modality).
A visual depitction of the breakdown of melodies from the two larger sections in terms of count data is presented IN FIGURE HERE.

\begin{itemize}
\tightlist
\item
  FIGURE HERE
\end{itemize}

In terms of analyzable data, the 783 melodies are all encoded in \texttt{kern} format with each file containing metadata listing the excerpt's LIST HERE.
Overall, the corpus consists of XXXXX digital tokens, a subset of which are XXX note heads.
All melodies in the corpus were encoded by hand by the author using MUSE SCORE 3, initially saved as XML, then converted to kern using the HUMDRUM EXTRAS \texttt{xml2hum} with the current meta data added using the \texttt{name-of-script.R} file.
Further addition to the metadata can be added with modifications to \texttt{name-of-scrpit.R}.

From a more meaningful point of view, the descriptive statistics of the corpus are displayed in FIGURE TWO and FIGURE THREE.

\begin{itemize}
\tightlist
\item
  FIGURE TWO (subset out Section Five )
\item
  FIGURE THREE (Section Five )
\end{itemize}

\hypertarget{comparison}{%
\section{Comparison}\label{comparison}}

Further descriptive statistics of the corpus generated from MULLENSIEFEN'S FANTASTIC TOOLBOX can help contextualize the \emph{MELOSOL} corpus in context with other corpora commonly used in the literature.
One of the most cited corpora in the field of computational musicology is ESSEN.
ESSEN contains XYZ and is often taken as proxy for representing implicit understanding via statistical learning (HURON) and PEARCE and OTHERS.
Essen also has Chinese songs.
The \emph{MeloSol} corpus also falls under umbrella of Western Music and as discussed below, may be helpful novel corpus for continuing to investigate claims.
Since publication of ESSEN there has also been DENSMORE CITE.
DENSMORE is collection of melodies encoded by Shanahan and Shanahan.
From musicological point of view, both DENSMORE and CHINESE are expected to be different for reasons of both location as well as style.
Here we compare the two to get high level reduction of idea.

First in FIGURE FOUR we compare high level descriptive statistics between WESTERN ESSEN, CHINA, DENSMORE, and MELOSOL.
Figure contains comparative overlap of LIST OF FEATURES HERE.
Note there is a very large difference in the size of MELOSOL (and others) compared to ESSEN.

\begin{itemize}
\tightlist
\item
  FIGURE FOUR
\end{itemize}

Second in FIGURE FIVE we compoare MORE ABSTRACT FEAGURES

\begin{itemize}
\tightlist
\item
  FIGURE FIVE
\end{itemize}

Here is some small comparison on the differences in features.

\hypertarget{useful}{%
\section{Useful}\label{useful}}

As the \emph{MeloSol} corpus is made of Western music, can be used to continue research that has made claims about certain features of Western music if need proxy.
For example there are a lot of claims made by HURON about contour class that have been initiall explored by BAKER.
There also have been lots of modeling of expeectation using IDyOM by Pearce that have used ESSEN.
If buy the idea of sample population as generation, this could be taken forward in that area.

Also note that now have dataset was initially generated using pedagogical materials and might be helfpul in that domain.
For example, extending work of MY DISSERATATION could look at proxies of difficulty using FANTASTIC.
Could also see if enough data here can be used for generative data analyses using LSTM.

\hypertarget{data-analysis}{%
\subsection{Data analysis}\label{data-analysis}}

We used R (Version 3.6.2; R Core Team, 2019) and the R-package \emph{papaja} (Version 0.1.0.9942; Aust \& Barth, 2020) for all our analyses.

\newpage

\hypertarget{references}{%
\section{References}\label{references}}

\begingroup
\setlength{\parindent}{-0.5in}
\setlength{\leftskip}{0.5in}

\hypertarget{refs}{}
\leavevmode\hypertarget{ref-R-papaja}{}%
Aust, F., \& Barth, M. (2020). \emph{papaja: Create APA manuscripts with R Markdown}. Retrieved from \url{https://github.com/crsh/papaja}

\leavevmode\hypertarget{ref-berkowitzNewApproachSight2011}{}%
Berkowitz, S., Fontrier, G., Kraft, L., Goldstein, P., \& Smaldone, E. (2011). \emph{A new approach to sight singing} (5th ed). New York: W.W. Norton.

\leavevmode\hypertarget{ref-R-base}{}%
R Core Team. (2019). \emph{R: A language and environment for statistical computing}. Vienna, Austria: R Foundation for Statistical Computing. Retrieved from \url{https://www.R-project.org/}

\endgroup

\end{document}
