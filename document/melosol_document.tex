\PassOptionsToPackage{unicode=true}{hyperref} % options for packages loaded elsewhere
\PassOptionsToPackage{hyphens}{url}
%
\documentclass[english,man]{apa6}
\usepackage{lmodern}
\usepackage{amssymb,amsmath}
\usepackage{ifxetex,ifluatex}
\usepackage{fixltx2e} % provides \textsubscript
\ifnum 0\ifxetex 1\fi\ifluatex 1\fi=0 % if pdftex
  \usepackage[T1]{fontenc}
  \usepackage[utf8]{inputenc}
  \usepackage{textcomp} % provides euro and other symbols
\else % if luatex or xelatex
  \usepackage{unicode-math}
  \defaultfontfeatures{Ligatures=TeX,Scale=MatchLowercase}
\fi
% use upquote if available, for straight quotes in verbatim environments
\IfFileExists{upquote.sty}{\usepackage{upquote}}{}
% use microtype if available
\IfFileExists{microtype.sty}{%
\usepackage[]{microtype}
\UseMicrotypeSet[protrusion]{basicmath} % disable protrusion for tt fonts
}{}
\IfFileExists{parskip.sty}{%
\usepackage{parskip}
}{% else
\setlength{\parindent}{0pt}
\setlength{\parskip}{6pt plus 2pt minus 1pt}
}
\usepackage{hyperref}
\hypersetup{
            pdftitle={The MeloSol Corpus},
            pdfkeywords={corpus studies, FAIR data, kern},
            pdfborder={0 0 0},
            breaklinks=true}
\urlstyle{same}  % don't use monospace font for urls
\usepackage{graphicx,grffile}
\makeatletter
\def\maxwidth{\ifdim\Gin@nat@width>\linewidth\linewidth\else\Gin@nat@width\fi}
\def\maxheight{\ifdim\Gin@nat@height>\textheight\textheight\else\Gin@nat@height\fi}
\makeatother
% Scale images if necessary, so that they will not overflow the page
% margins by default, and it is still possible to overwrite the defaults
% using explicit options in \includegraphics[width, height, ...]{}
\setkeys{Gin}{width=\maxwidth,height=\maxheight,keepaspectratio}
\setlength{\emergencystretch}{3em}  % prevent overfull lines
\providecommand{\tightlist}{%
  \setlength{\itemsep}{0pt}\setlength{\parskip}{0pt}}
\setcounter{secnumdepth}{0}

% set default figure placement to htbp
\makeatletter
\def\fps@figure{htbp}
\makeatother

% Manuscript styling
\usepackage{upgreek}
\captionsetup{font=singlespacing,justification=justified}

% Table formatting
\usepackage{longtable}
\usepackage{lscape}
% \usepackage[counterclockwise]{rotating}   % Landscape page setup for large tables
\usepackage{multirow}		% Table styling
\usepackage{tabularx}		% Control Column width
\usepackage[flushleft]{threeparttable}	% Allows for three part tables with a specified notes section
\usepackage{threeparttablex}            % Lets threeparttable work with longtable

% Create new environments so endfloat can handle them
% \newenvironment{ltable}
%   {\begin{landscape}\begin{center}\begin{threeparttable}}
%   {\end{threeparttable}\end{center}\end{landscape}}
\newenvironment{lltable}{\begin{landscape}\begin{center}\begin{ThreePartTable}}{\end{ThreePartTable}\end{center}\end{landscape}}

% Enables adjusting longtable caption width to table width
% Solution found at http://golatex.de/longtable-mit-caption-so-breit-wie-die-tabelle-t15767.html
\makeatletter
\newcommand\LastLTentrywidth{1em}
\newlength\longtablewidth
\setlength{\longtablewidth}{1in}
\newcommand{\getlongtablewidth}{\begingroup \ifcsname LT@\roman{LT@tables}\endcsname \global\longtablewidth=0pt \renewcommand{\LT@entry}[2]{\global\advance\longtablewidth by ##2\relax\gdef\LastLTentrywidth{##2}}\@nameuse{LT@\roman{LT@tables}} \fi \endgroup}

% \setlength{\parindent}{0.5in}
% \setlength{\parskip}{0pt plus 0pt minus 0pt}

% \usepackage{etoolbox}
\makeatletter
\patchcmd{\HyOrg@maketitle}
  {\section{\normalfont\normalsize\abstractname}}
  {\section*{\normalfont\normalsize\abstractname}}
  {}{\typeout{Failed to patch abstract.}}
\makeatother
\shorttitle{MeloSol}
\author{David John Baker\textsuperscript{1}}
\affiliation{
\vspace{0.5cm}
\textsuperscript{1} Louisiana State University}
\authornote{David John Baker now works at Flatiron School in London, England.


Correspondence concerning this article should be addressed to David John Baker, . E-mail: davidjohnbaker1@gmail.com}
\keywords{corpus studies, FAIR data, kern\newline\indent Word count: X}
\DeclareDelayedFloatFlavor{ThreePartTable}{table}
\DeclareDelayedFloatFlavor{lltable}{table}
\DeclareDelayedFloatFlavor*{longtable}{table}
\makeatletter
\renewcommand{\efloat@iwrite}[1]{\immediate\expandafter\protected@write\csname efloat@post#1\endcsname{}}
\makeatother
\usepackage{lineno}

\linenumbers
\usepackage{csquotes}
\ifnum 0\ifxetex 1\fi\ifluatex 1\fi=0 % if pdftex
  \usepackage[shorthands=off,main=english]{babel}
\else
  % load polyglossia as late as possible as it *could* call bidi if RTL lang (e.g. Hebrew or Arabic)
  \usepackage{polyglossia}
  \setmainlanguage[]{english}
\fi

\title{The MeloSol Corpus}

\date{}

\abstract{
This paper introduces the \emph{MeloSol} corpus, a collection of 783 Western, tonal monophonic melodies. We first begin by describing the overal structure of the corpus, then proceede to detail its contents as they would be helpful for researchers working in the field of computational musicology or music psychology. In order to contextualizs the MeloSol corpus, compare descriptive statistics generated using the FANTASTIC feature extraction toolkit with that of the Essen Folk Song Collection as well as The Densmore Collection of Native American Songs. We suggest posible uses of this corpus including extending research which investigates Western tonality, perceptual experiments neededing novel ecological stimuli, or work involving the musical generation of monophonic melodies in the style of Western tonal.
}

\begin{document}
\maketitle

\hypertarget{introdution}{%
\section{Introdution}\label{introdution}}

This data report introduces the \emph{MeloSol} corpus, a collection of 783 monophonic melodies taken from \emph{A New Approach to Sight Singing: Fifth Edition} (Berkowitz, Fontrier, Kraft, Goldstein, \& Smaldone, 2011).
The title \emph{MeloSol} derives from a combination of the corpus' content-- \emph{Melo} dic data-- and the first name of the original author of the collection, \emph{Sol} Berkowitz.

The corpus is divided into two major sections: a collection of sight singing melodies composed specifically for pedagogical purposes (n = 629) taken from Chapter One and examples from Western Classical literature (n = 154) taken from Chapter Five.
The original text also contains materials for practicing rhythm (Chapter Two), Singing Duets (Chapter Three), Sing and Plays that incoproate a melody and piano accompaniament (Chapter Four), and Supplementary Exercises that are not included here.
Within each of the larger sections exists five further subdivisions.
These five subdivisions are mapped in conjunction with the trajectory of many aural skills classrooms.

For example, the first section of both the sight singing melodies and the first section of literature align with melodies a first semester undergraduate student in a music degree program might be expected to learn during their first semester of university in an aural skills classroom.
As the original book was designed as a pedaogical text, each section of the book and consequently each melody within each section is meant to increase in complexity as new topics are being introduced.
The fifth and final section of both the sight singing melodies and examples from the literature contains melodies which break from Western tonal practice.
These melodies contain either modal, atonal, or tonally ambigious melodies.
A visual depitction of the breakdown of melodies from the two larger sections in terms of count data is presented in FIGURE ONE.

\begin{itemize}
\tightlist
\item
  FIGURE ONE HERE
\end{itemize}

In terms of analyzable data, the 783 melodies are encoded in \texttt{**kern} format (Huron, 1994), with each individual file containing metadata listing the unique identifier, chapter from which the melody originates, section within that chapter of the larger text, page number, as well as what mode the encoder labeled the melody as.
Modes were only noted for a small subset of the corpus, the vast majority of these melodies are either major (ionian) or minor (aeolean).
Other corpora should be consulted for questions pertaining to mode such as work by Albrecht and Huron (Albrecht \& Huron, 2014).

Overall, the corpus consists of 49,730 tokens, a subset of which are 36,641 note heads.
All melodies in the corpus were encoded by hand using the software MuseScore (Werner, Nicholas, \& Bonte, 2019), initially saved as XML, then converted to \texttt{**kern} using the humdrum extras \texttt{xml2hum} tool (Sapp, 2008) with the current meta data added using the \texttt{name-of-script.R}.
Further addition to the metadata can be added with modifications to \texttt{name-of-script} found in the \texttt{scripts} directory.
We describe the corpus from a macro perspective in figure TWO and THREE.
Figure TWO presents the combined tonal materials, sections one through four of Chapter One and Five; figure THREE presents the same information for section five of Chapter One and Five.
Tonal and non-tonal materials were separated as not to distort key representations since non-tonal melodies are encoded with a no sharps, no flats key signature.

\begin{itemize}
\tightlist
\item
  FIGURE TWO (Subset out Section Five)
\item
  FIGURE THREE (Section Five)
\end{itemize}

\hypertarget{comparison}{%
\section{Comparison}\label{comparison}}

In order to further contextualize the \emph{MeloSol} corpus with the context of other corpora found in the literature, we briefly compare descriptive statistics from the \emph{MeloSol} corpus with both \emph{The Densmore Collection of Native American Songs} (Neubarth, Shanahan, \& Conklin, 2018; Shanahan \& Shanahan, 2014) as well as the European and Chinese subset of the \emph{Essen Folk Song Collection} (Schaffrath, 1995).
We chose both the \emph{Densmore} as well as the \emph{Essen} collection as both corpora represent corpora of melodies that have been used to study singing as well as investigations into computationally modeling the implicit understanding of patterns found in different musical cultures.
Further, we compare the \emph{MeloSol} with the \emph{Essen} collection as the \emph{Essen} collection has been used as a proxy for representing the implicit understanding of the structure of Western, tonal music in computational models that depend theoretically on the concept of implicit, statistical learning (Demorest \& Morrison, 2016; Huron, 2006; Pearce, 2018).
Comparisons of descriptive statistics were conducted using the FANTASTIC toolbox (Mullensiefen, 2009).
The accompanying calculations for each melody are found in \texttt{corpus/melosol\_fantastic\_features.csv}.

In FIGURE FOUR, we compare \ldots{}
In FIGURE FIVE, we compare \ldots{}

\begin{itemize}
\tightlist
\item
  FIGURE FOUR
\item
  FIGURE FIVE
\end{itemize}

\hypertarget{useful}{%
\section{Useful}\label{useful}}

As the \emph{MeloSol} corpus comprises Western, tonal music, this corpora might be utalized in order to continue research investigating empirical claims about about patterns intrinsic to Western, tonal music.
For example claims made by Huron CITE regarding contour class-- initially explored using this dataset by BAKER-- could be further modeled using \emph{MeloSol}.
Additionally, as \emph{MeloSol} strictly contains music associated with Western, tonal music, the corpus could be used in further work replacing the \emph{Essen} collection as a dataset in which to train computational models of melodic expectation (Pearce, 2018).
We finally note that as this corpus was initially developed in order to investigate how to make pedagogical improvements in aural skills classrooms, using \emph{MeloSol} for this purpose would be a logical extension to this programme of research CITE ME.

\hypertarget{data-analysis}{%
\subsection{Data analysis}\label{data-analysis}}

We used R (Version 3.6.2; R Core Team, 2019) and the R-package \emph{papaja} (Version 0.1.0.9942; Aust \& Barth, 2020) for all our analyses.

\newpage

\hypertarget{references}{%
\section{References}\label{references}}

\begingroup
\setlength{\parindent}{-0.5in}
\setlength{\leftskip}{0.5in}

\hypertarget{refs}{}
\leavevmode\hypertarget{ref-albrecht2014statistical}{}%
Albrecht, J. D., \& Huron, D. (2014). A statistical approach to tracing the historical development of major and minor pitch distributions, 1400-1750. \emph{Music Perception: An Interdisciplinary Journal}, \emph{31}(3), 223--243.

\leavevmode\hypertarget{ref-R-papaja}{}%
Aust, F., \& Barth, M. (2020). \emph{papaja: Create APA manuscripts with R Markdown}. Retrieved from \url{https://github.com/crsh/papaja}

\leavevmode\hypertarget{ref-berkowitzNewApproachSight2011}{}%
Berkowitz, S., Fontrier, G., Kraft, L., Goldstein, P., \& Smaldone, E. (2011). \emph{A new approach to sight singing} (5th ed). New York: W.W. Norton.

\leavevmode\hypertarget{ref-demorest201612}{}%
Demorest, S. M., \& Morrison, S. J. (2016). 12 quantifying culture: The cultural distance hypothesis of melodic expectancy. \emph{The Oxford Handbook of Cultural Neuroscience}, 183.

\leavevmode\hypertarget{ref-huronHumdrumToolkitReference1994}{}%
Huron, D. (1994). The Humdrum Toolkit: Reference Manual. Center for Computer Assisted Research in the Humanities.

\leavevmode\hypertarget{ref-huronSweetAnticipation2006}{}%
Huron, D. (2006). \emph{Sweet Anticipation}. MIT Press.

\leavevmode\hypertarget{ref-mullensiefenFantasticFeatureANalysis2009}{}%
Mullensiefen, D. (2009). Fantastic: Feature ANalysis Technology Accessing STatistics (In a Corpus): Technical Report v1.5.

\leavevmode\hypertarget{ref-neubarthSupervisedDescriptivePattern2018}{}%
Neubarth, K., Shanahan, D., \& Conklin, D. (2018). Supervised descriptive pattern discovery in Native American music. \emph{Journal of New Music Research}, \emph{47}(1), 1--16. \url{https://doi.org/10.1080/09298215.2017.1353637}

\leavevmode\hypertarget{ref-pearceStatisticalLearningProbabilistic2018a}{}%
Pearce, M. T. (2018). Statistical learning and probabilistic prediction in music cognition: Mechanisms of stylistic enculturation: Enculturation: Statistical learning and prediction. \emph{Annals of the New York Academy of Sciences}, \emph{1423}(1), 378--395. \url{https://doi.org/10.1111/nyas.13654}

\leavevmode\hypertarget{ref-R-base}{}%
R Core Team. (2019). \emph{R: A language and environment for statistical computing}. Vienna, Austria: R Foundation for Statistical Computing. Retrieved from \url{https://www.R-project.org/}

\leavevmode\hypertarget{ref-sappHumdrumExtras2008}{}%
Sapp, C. (2008). Humdrum Extras.

\leavevmode\hypertarget{ref-schaffrathEssenFolkSong1995}{}%
Schaffrath, H. (1995). The Essen Folk Song Collection, D. Huron.

\leavevmode\hypertarget{ref-shanahanDensmoreCollectionNative2014}{}%
Shanahan, D., \& Shanahan, E. (2014). The Densmore Collection of Native American Songs: A New Corpus for Studies of Effects of Geography, Language, and Social Function on Folk Song. In \emph{Proceedings of the Fourteenth Annual International Conference for Music Perception and Cognition}. San Francisco.

\leavevmode\hypertarget{ref-wernerMuseScore2019}{}%
Werner, S., Nicholas, F., \& Bonte, T. (2019). MuseScore.

\endgroup

\end{document}
